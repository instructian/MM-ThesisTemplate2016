%
%
% click on the rich text option above to read this in an easier view
% DO NOT EDIT THIS
%

\textbf{Requirements for Written Documentation of Thesis Project
}
\textbf{The document should be 8000 to 20,000 words plus any appropriate images and diagrams.}

The rough time line is:
\textbf{End of Fall Quarter} - submit draft of:  Original Concept, Prior Art, a bibliography in progress
\textbf{End of Winter Quarter }- Submit draft of Detailed Project Development & Methods and User Testing
\textbf{End of Spring Quarter - See below}
\textbf{First week of May} – near final draft due to all Graduate Committee members
\textbf{Third week of May }– feedback from Committee members due
\textbf{First week of June} – final draft due to Committee
\textbf{Second week of June} – if final draft is acceptable it will be signed by the Committee. If the draft needs a revision, feedback will be provided.
\textbf{Third week of June} – revised final draft due (if required)

The following structure is recommended:

\chapter{Title}

\section{Abstract}
A good abstract explains in one line why the paper is important. Then it describes your starting point (your thesis question), and goes on to give a summary of your major results. The final sentences explain the major implications of your work. A good abstract is concise, readable, and precise.
\begin{itemize}
\item Length should be ~ 1-2 paragraphs, approx. 400 words.
\item Abstracts generally do not have citations.
\item Information in title should not be repeated.
\item Be explicit. Don’t make vague statements.
\item Answers to these questions should be found in the abstract:
\begin{enumerate}
\item What did you do?
\item Why did you do it? What question were you trying to answer?
\item How did you do it? State methods.
\item What did you learn? State major results.
\item Why does it matter? Point out at least one significant implication.
\end{enumerate}
\end{itemize}

\section{Keywords}
List keywords that would aid in searching for your project.

\section{Table of Contents}

\section{Original Concept}
Explain initial concept. Include a narrative about how you chose your methods.

\section{Prior Art}
An overview of prior art. In a science paper, this would be a literature review. What other similar projects have already been done? What is the state of the art in the area of your research? Art is considered in a wide spectrum from commercial projects to fine art ones. In masters thesis projects the goals is to advance the state of the art. The paper is written with this in mind, so that those interested in exploring this field can benefit from and build upon the work described within. It must give enough details to make this possible.

\section{Detailed Project Development & Methods}
\subsection{Content}
\subsection{Technology}
\subsubsection{Code}
\subsubsection{Graphics/Video}
\subsubsection{Projection and Physical Space}
\subsubsection{History of development of Content and Technology} (eg history of research -- what options were tested and discarded? why? Don’t cover every minor twist, turn or drama -- only include lessons learned that others could benefit from)

\section{User Testing}
How were the user testing questions developed? How is the user experience evaluated?

\section{Results}
Describe results thoroughly. Report raw results, not interpretation. Results will include what you got to work, what the results of user testing were, and, to the extent that you used an iterative process for development, the specific results of the process. Good results (ie, those that support your thesis question) are OK. Bad results are OK. Mixed results are OK. But vague or unclear results, no results, or mushy waffle instead of results is not.

\section{Discussion}
This is where you interpret the results. What was successful and why. For negative & ambiguous results - analyze. Are there problems in the testing instrument? With the methods? With the thesis question? Examine the final aesthetics? Describe the process of testing and fine tuning. The interpretation is addressed to people who might want to build upon the work, those interested in this area of multimedia.

\section{Dissemination}
How the work was promoted, explained, and presented. Include details of website, release schedule, venues (iTunes Store, Maker Faire, MM Grad Thesis presentation in June, etc), promotion efforts (press releases, kickstarter campaigns, etc), published papers (this or spinoffs), open-source software (libraries or apps) contributed, conferences or panels, etc.

\section{Conclusion}
• What is the strongest and most important statement that you can make from your 
observations?
• If you met the reader at a meeting six months from now, what do you want them to remember about your paper?
• Refer back to problem posed, and describe the conclusions that you reached from
carrying out this investigation, summarize new observations, new interpretations, and new insights that have resulted from the present work.
• Include the broader implications of your results.
• Do not repeat word for word the abstract, introduction or discussion.

\section{Recommendations}
Goal is to help people build upon this work.
• Further research to fill in gaps in our understanding.
• Directions for future investigations on this or related topics.

\section{Acknowledgments}

\section{Bibliography: include only what was cited and relevant.}

\section{Appendix A: Project Files = a CD with files including all code, graphics, video.}

\section{Appendix B (optional): Production Logs, User testing instruments (questionnaires, notes, etc)}
Blog could be included as an appendix.